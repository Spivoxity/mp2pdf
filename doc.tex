\input jmsmac
\stretchpage

\chapter* Using MetaPost with pdfTeX/Mike Spivey, April 2009.

Producing PDF files with pdfTeX is more convenient than the
combination of ordinary TeX with dvips and ps2pdf, and it's nice to be
able to use TrueType fonts, but MetaPost produces its output as
Encapsulated Postscript, and pdfTeX is not directly able to
incorporate this into a document.

Several solutions to this problem are possible, and Hans Hagen (as
a kind of {\it tour de force\/}) has implemented TeX macros that are
capable of parsing the restricted form of Postscript output by
MetaPost.  My solution is a separate program, implemented with more
traditional software tools, that takes MetaPost output and translates
it into TeX {\it input\/} for pdfTeX, using the \verb+\pdfliteral+
primitive to write raw PDF code.  This approach has a number of
advantages, including robustness and freedom from the limited
precision of TeX arithmetic.  Using pdfTeX itself to write the PDF
output makes the program simpler and more portable.

\section Workflow.

With \verb/mp2pdf/, figures are described in the usual MetaPost
language, and MetaPost translates them into EPS form.  The
\verb/mp2pdf/ program parses the subset of Postscript that is output
by MetaPost, and produces a file of TeX input that, when processed by
pdfTeX, results in PDF output with the same meaning as the EPS.  A
single run of pdfTeX the combines the text of the containing document
with the TeX code for the figures to produce a PDF file as output.

If the figures use the \verb/btex/ \dots \verb/etex/ feature of
MetaPost for labels, then ordinary (non-PDF) TeX will be invoked by
MetaPost to typeset the labels.  The resulting DVI file is processed
behind the scenes with dvimp to produce MetaPost input, and this
contributes to the Postscript output of MetaPost, and is subsequently
translated by mp2pdf into TeX input once more.  The text then passes
through TeX again (pdfTeX this time) and becomes part of the final PDF
output.  As we'll see, it's important that whatever typeset result was
produced by the first pass through TeX is not disturbed by passing it
through TeX the second time, but also important that pdfTeX is able to
take account in compiling font subsets of the glyphs that are used in
the labels.

\section Dealing with text.

\resetpar

\input minus-1.mtx

\font\ttx=lbtr at10pt

\def\verbfont{\tt}

?`Ol\'e?

{\tt ?`Ol\'e?}

{\ttx ?`Ol\'e?}

\verb/?`/ ?`

\bye
