% mp2pdf.tex

% This file contains optional macros for scaling mp2pdf output to
% a specified width or height.  Say
%
%     \mpwidth=100pt \input fig.mtx
%
% to scale the figure so that its width is exactly 100pt. Similarly
% for \mpheight.  The constuction
%
%     \def\mpscale{0.5} \input fig.mtx
%
% works with or without these macros to scale the figure by a
% specified linear factor.

\newdimen\mpdimen

\def\mpwidth{\let\mpsetsize=\mpsetwidth\mpdimen}
\def\mpheight{\let\mpsetsize=\mpsetheight\mpdimen}

\def\mpsetwidth{\mpdivide0 \dimen0=\mpdimen \dimen1=\mpscale\dimen1 }
\def\mpsetheight{\mpdivide1 \dimen0=\mpscale\dimen0 \dimen1=\mpdimen}

% This division routine computes x = 65536 (a/b) = (a/b) * 1pt
% where a = \mpdimen, b = \dimen#1. It is an unrolled loop with the invariant
%     0 <= y < b  and  65536 (a/b) = x + z (y/b), 
% where z takes the values 65536, 256, 1.

% \mpdivide -- define \mpscale to be the value of \mpdimen / \dimen#1
\def\mpdivide#1{%
  % We are inside a group, so let's take advantage of that.
  \dimendef\b=#1 \dimendef\x=2 \dimendef\y=3 \dimendef\t=4
  \y=\mpdimen \t=\y \divide\t by\b \x=65536\t 
  \multiply\t by\b \advance\y by-\t 
  \multiply\y by256 \t=\y \divide\t by\b \advance\x by256\t 
  \multiply\t by\b \advance\y by-\t
  \multiply\y by256 \divide\y by\b \advance\x by\y
  \edef\mpscale{\expandafter\xyzzy\the\x}}

{\catcode`p=12 \catcode`t=12 \gdef\xyzzy#1pt{#1}}
